% layout
\usepackage{geometry}
\usepackage{indentfirst}
\usepackage{setspace}
\usepackage{titlesec}
\usepackage[subfigure]{tocloft}

% citations
\usepackage{natbib}
\usepackage{apalike}
%\usepackage{etoolbox}
%\apptocmd{\sloppy}{\hbadness 10000\relax}{}{}

% include citations inline
\usepackage{bibentry}
\nobibliography*

% table
\usepackage{threeparttable}
\usepackage{multirow}

% figures
\usepackage{graphicx}
\usepackage{booktabs}
\usepackage{multicol}
\usepackage{listings}
\usepackage{subfig}
\usepackage{overpic}


% algorithm
\usepackage{algorithm}
%\usepackage{algpseudocode}
\usepackage{algorithmic}

% math
\usepackage{amsthm}
\usepackage{amsmath}
\usepackage{amssymb}
\usepackage{units}
\usepackage{array}

% typography
\usepackage{times}
\usepackage{microtype}
\usepackage{textcomp}

% macro support
\usepackage{xspace}
\usepackage[table]{xcolor}
\usepackage{pdflscape}
\usepackage{rotating}


% page floats
\usepackage{afterpage}

% pdf links
%\usepackage[hidelinks]{hyperref}

% use proper margins.
\geometry{letterpaper,left=1.25in,top=1in,right=1.25in,bottom=1in,nohead}

% double-space text
\doublespacing

% color
\usepackage{color}

% Center chapter titles, omit page numbers.
\titleformat{\chapter}[display]{\fillast\bfseries}{\Large\MakeUppercase{\chaptertitlename} \thechapter}{-11pt}{\huge\singlespacing}[\thispagestyle{empty}]

% extend to 2in top margins, leave 22pts = 2x font size after heading
\titlespacing{\chapter}{0in}{0.62in}{22pt}

% indent paragraphs four spaces throughout the thesis/dissertation.
\setlength{\parindent}{4ex}

% tweak spacing of paragraph labels.
\titlespacing{\paragraph}{0in}{0.08in}{0.07in}

% we want numbered subsubsections
\setcounter{secnumdepth}{3}
\setcounter{tocdepth}{3}

% we need to double-space between footnotes.
\setlength{\footnotesep}{13pt}

% we don't want crazy vertical spacing.
\raggedbottom

% we don't want abandoned words.
\clubpenalty=10000 
\widowpenalty=10000

% cite with (name, year)
\renewcommand{\cite}{\citep}

% common abbreviations
\newcommand{\eg}{{\it e.g.}\xspace}
\newcommand{\ie}{{\it i.e.}\xspace}
\newcommand{\etc}{{\it etc.}\xspace}
\newcommand{\etal}{\emph{et~al}\mbox{.}\xspace}

\newcommand{\xth}{\ensuremath{^{\text{th}}}\xspace}
%\newcommand{\fst}{\ensuremath{^{\text{st}}}\xspace}

% common math notation
\newcommand{\NAT}[0]{\mathbb{N}\xspace}
\newcommand{\fun}[1]{\mathit{#1}} % typeset as function name
\newcommand{\setsize}[1]{\left| #1 \right|}
\newcommand{\setdef}[2]{\left\{ #1 \ \left|\  #2\right.\right\}}
\newcommand{\dispsum}[0]{\displaystyle\sum}

\newcommand{\defeq}[0]{\triangleq}
\renewcommand{\mod}{\operatorname{mod}}

% time units
\newcommand{\mus}[0]{\ensuremath{\mu s}\xspace}
\newcommand{\us}[0]{\ensuremath{\mu s}}
\newcommand{\ms}[0]{\ensuremath{\fun{ms}}\xspace}

% algorithm names
\newcommand{\kwfont}[1]{\textsf{#1}\xspace} %\small
% variable name
\newcommand{\var}[1]{\ensuremath{{\fun{#1}}}\xspace} %\small

%http://hstuart.dk/2007/08/03/programming-latex-%E2%80%94-writing-commands/
\newcommand{\mkkw}[2]{\newcommand{#1}[0]{\kwfont{#2}}}

% fancy symbols and functions
\newcommand{\Alg}[0]{{\mathcal A}}
\newcommand{\Test}[0]{{\mathcal T}}
\newcommand{\Mach}[0]{{\mathcal M}}

\newcommand{\usum}[0]{u_{\mathrm{sum}}}
\newcommand{\umax}[0]{u_{\mathrm{max}}}
\newcommand{\umin}[0]{u_{\mathrm{min}}}
\newcommand{\utop}[0]{u_{\mathrm{top}}}

\newcommand{\esum}[0]{e_{\mathrm{sum}}}
\newcommand{\emax}[0]{e_{\mathrm{max}}}
\newcommand{\emin}[0]{e_{\mathrm{min}}}
\newcommand{\etop}[0]{e_{\mathrm{top}}}

\newcommand{\dsum}[0]{\delta_{\mathrm{sum}}}
\newcommand{\dmax}[0]{\delta_{\mathrm{max}}}
\newcommand{\dmin}[0]{\delta_{\mathrm{min}}}
\newcommand{\dtop}[0]{\delta_{\mathrm{top}}}

\newcommand{\prio}[0]{\mathsf Y}
\newcommand{\eprio}[0]{\mathsf y}

\newcommand{\Tr}{\operatorname{Tr}}

\newcommand{\TODO}[1]{\textbf{\textcolor{red}{[TODO: #1]}}}

% src code
\newcommand{\src}[1]{\textsf{\small #1}\xspace}

% references
\newcommand{\chref}[1]{Chapter~\ref{ch:#1}\xspace}
\newcommand{\chrefs}[2]{Chapters~\ref{ch:#1} and~\ref{ch:#2}\xspace}
\newcommand{\secref}[1]{Section~\ref{sec:#1}\xspace}
\newcommand{\figref}[1]{Figure~\ref{fig:#1}\xspace}
\newcommand{\figrefi}[2]{Figure~\ref{fig:#1}(#2)\xspace}
\newcommand{\tabref}[1]{Table~\ref{tab:#1}\xspace}
\newcommand{\lemref}[1]{Lemma~\ref{lem:#1}\xspace}
\newcommand{\thmref}[1]{Theorem~\ref{thm:#1}\xspace}
\newcommand{\defref}[1]{Definition~\ref{def:#1}\xspace}
\newcommand{\exref}[1]{Example~\ref{ex:#1}\xspace}
\newcommand{\equref}[1]{Equation~(\ref{eq:#1})\xspace}
\newcommand{\inequref}[1]{Inequality~(\ref{eq:#1})\xspace}
\newcommand{\lstref}[1]{Listing~\ref{lst:#1}\xspace}
\newcommand{\pref}[1]{page~\pageref{p:#1}\xspace}





% special footnotes
% from http://help-csli.stanford.edu/tex/latex-footnotes.shtml
\long\def\symbolfootnote[#1]#2{\begingroup%
\def\thefootnote{\fnsymbol{footnote}}\footnote[#1]{#2}\endgroup}

% theorems, etc.
\newtheoremstyle{mylemthm}% hnamei 
        {6pt}% hSpace abovei 
        {3pt}% hSpace belowi 
        {\slshape}% hBody fonti 
        {}% hIndent amounti1
        {\bfseries}% hTheorem head fonti 
        {.}% hPunctuation after theorem headi 
        {.5em}% hSpace after theorem headi2
        {}% hTheorem head spec (can be left empty, meaning `normal')i

\theoremstyle{mylemthm}
\newtheorem{theorem}{Theorem}[chapter]
\newtheorem{lemma}{Lemma}[chapter]

\newtheoremstyle{mydef}% hnamei 
        {3pt}% hSpace abovei 
        {3pt}% hSpace belowi 
        {\normalfont}% hBody fonti 
        {}% hIndent amounti1
        {\bfseries}% hTheorem head fonti 
        {.}% hPunctuation after theorem headi 
        {.5em}% hSpace after theorem headi2
        {\thmname{#1} \thmnumber{#2}\thmnote{#3}}% hTheorem head spec (can be left empty, meaning `normal')i
\theoremstyle{mydef}

%% Flush words right at end of paragraph.
%% From: http://tex.stackexchange.com/questions/16330/hfill-after-linebreak
\newcommand\rightparend[1]{{%
      \unskip\nobreak\hfil\penalty50
      \hskip2em\hbox{}\nobreak\hfil\textbf{#1}%
      \parfillskip=0pt \finalhyphendemerits=0 \par}}

\newtheorem{definition}{Definition}[chapter]
\newtheorem{xxexample}{Example}[chapter]

\makeatletter
\renewcommand{\maketag@@@}[1]{\hbox{\m@th\normalsize\normalfont#1}}%
\makeatother

%% "inherent" from xxexample, but place box at the end of example.
\newenvironment{example}{
\begin{xxexample}
}{
\rightparend{$\Diamond$}
\end{xxexample}
}
% \qed   \sqbullet \blackdiamond \vartriangleleft

\graphicspath{{figures/}}














\newcommand{\trace}{\operatorname{Tr}}
\newcommand{\vectorize}{\operatorname{vec}}
\newcommand{\sign}{\operatorname{sgn}}
\newcommand{\cdf}{\operatorname{cdf}}
\newcommand{\argmax}{\operatorname{argmax}}




\newcommand{\Cfree}{\ensuremath{\mathcal C_{\text{free}}}}
\newcommand{\Cobs}{\ensuremath{\mathcal C_{\text{obs}}}}
\newcommand{\Cspace}{\ensuremath{\mathcal C\text{-space}}}
\newcommand{\Ccont}{\ensuremath{\mathcal C_\text{cont}}}

\newcommand{\Rn}{\ensuremath{\mathbb R^n}}
\newcommand{\Rsqr}{\ensuremath{\mathbb R^2}}
\newcommand{\Rcubic}{\ensuremath{\mathbb R^3}}
\newcommand{\SEcubic}{\ensuremath{\mathbb {SE}(3)}}
\newcommand{\SEsqr}{\ensuremath{\mathbb {SE}(2)}}
\newcommand{\q}{\ensuremath{\mathbf q}}
\newcommand{\x}{\ensuremath{\mathbf x}}
\newcommand{\knn}{\ensuremath{k\text{-NN}}}


\newcommand{\LCS}{\ensuremath{LCS}}
\newcommand{\LCSa}{\ensuremath{LCS_0}}
\newcommand{\LCSb}{\ensuremath{LCS_1}}
\newcommand{\LCSi}{\ensuremath{LCS_i}}
\newcommand{\LCSiplus}{\ensuremath{LCS_{i+1}}}


\newcommand{\qv}{\ensuremath{\mathbf q_v}}
\newcommand{\qc}{\ensuremath{\mathbf q_c}}
\newcommand{\qs}{\ensuremath{\mathbf q_s}}
\newcommand{\qa}{\ensuremath{\mathbf q_a}}

\newcommand{\dist}{\operatorname{dist}}
\newcommand{\argmin}{\operatorname{argmin}}
\newcommand{\PDt}{\ensuremath{PD_t}}
\newcommand{\PDg}{\ensuremath{PD_g}}

\newcommand{\V}{\ensuremath{V}}
\newcommand{\SV}{\ensuremath{S}}
\newcommand{\SVLCSa}{\ensuremath{S_{LCS_0}}}
\newcommand{\SVLCS}{\ensuremath{S_{LCS}}}

\DeclareMathOperator{\isLeaf}{isLeaf}
\DeclareMathOperator{\exactIntersect}{exactIntersect}
\DeclareMathOperator{\overlap}{overlap}
\DeclareMathOperator{\pop}{pop}

\newcommand{\erf}{\operatornamewithlimits{cdf}}
\newcommand{\diag}{\operatornamewithlimits{diag}}

\renewcommand{\ttdefault}{lmtt}


\newenvironment{remark}[1][Remark]{\begin{trivlist}
\item[\hskip \labelsep {\bfseries #1}]}{\end{trivlist}}


