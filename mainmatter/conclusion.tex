\chapter{Conclusions and Future Work}
\label{chp:Conclusion}
In this dissertation, we have addressed a variety of computational challenges related to configuration spaces, including configuration space construction, efficient optimization in configuration space, and modeling uncertainty in configuration space. As research in configuration space continues, we expect progress in all these areas and more. While there is always more to do, the work presented in this dissertation has addressed many of the important issues in this field.

To summarize the main results presented in this dissertation:
\begin{description}
\item[Configuration Space Construction using Active Learning] We presented a novel approach to the approximation of configuration spaces. The main idea is to sample the configuration space and approximate the contact space based on machine learning classifiers, in particular support vector machines.
Furthermore, we use active learning techniques to select the samples during precomputation. Additionally, we use the precomputed configuration space for efficiently approximating the global penetration depth between two rigid objects.
\item[Configuration Space Construction using Instance-based Learning] We used instance-based learning to improve the performance of sample-based motion planners. The basic idea is to store the prior collision results as an approximate representation of $\Cfree$ and $\Cobs$, and to replace the expensive exact collision detection query by a relatively cheap probabilistic collision query. We integrated approximate collision routines with various sample-based motion planners and observe $30-100\%$ speedup on rigid and articulated robots, by enabling the robots to learn from their past experience.
\item[Parallel Motion Planning Framework] We introduced a whole motion planning algorithm on GPUs. Our algorithm can exploit all the parallelism within the PRM algorithm, including the high-level parallelism provided by the PRM framework and the low-level parallelism within different components of the PRM algorithm, such as collision detection and graph search. This makes our work the first to perform real-time motion planning and global navigation in general environments using GPUs.
\item[Parallel Collision Detection] We introduced two novel parallel collision query algorithms for real-time motion planning on GPUs. The first algorithm is based on configuration-packet tracing, is easy to implement, and can improve parallel performance by performing more coherent traversals and reducing the memory consumed by traversal stacks. The second algorithm is based on workload balancing, and decomposes parallel collision queries into fine-grained tasks. The algorithm uses a light-weight task-balancing strategy to guarantee that all GPU cores are fully utilized and achieves close to peak performance on GPUs.
\item[Parallel $k$-Nearest Neighbor] We presented an efficient GPU-based parallel Bi-level LSH algorithm to perform approximate $k$-nearest neighbor search in high-dimensional spaces. The Bi-level scheme can provide $k$-nearest neighbor results with higher quality than previous methods. In addition, our parallel algorithm provides more than a 40-fold acceleration over using LSH algorithms on CPUs.
\item[Proximity Computation for Noisy Geometry] We presented a novel and robust method for contact computation between noisy point cloud data using machine learning methods. We reformulate collision detection as a two-class classification problem and compute the collision probability at each point using support vector machines. This algorithm can be accelerated by using bounding volume hierarchies and performing a stochastic traversal.
\item[Proximity Computation for Noisy Geometry Streams] We presented two approaches for efficiently performing collision and distance queries on sensor data. The first method amortizes the sensor data pre-processing overhead over all the queries, and is suitable for static or simple environments. The second method shortens the traditional pipeline by directly performing queries between the robot links and an octree representing the environment. This approach completely avoids the data pre-processing overhead, and is suitable for dynamic or complex environments. Additionally, we combine this method with active sensing to improve robot safeness in uncertain or dynamic environments.
\end{description}

\section{Limitations and Future Work}
Our work has some limitations that could be addressed by future work.
\begin{description}
\item[Configuration Space Construction using Active Learning]
The accuracy and running time of our learning-based configuration space construction algorithm is a function of the combinatorial complexity of the contact
space and the sampling scheme. It is possible that our method may not generate a sufficient number of samples in small,
isolated components of contact space, or may take a high number of iterations.
The overall approach is probabilistic, and all our error bounds are derived in terms of expected error.
For future work, the basic components of our method, such as SVM learning and collision detection, can be accelerated using GPU parallelism. We can use other active learning techniques to improve the sampling, as well as other classifiers or learning techniques to improve the accuracy or convergence of the approximate contact space. It would be useful to derive tight theoretical error bounds for active learning algorithms based on exploitation and exploration. It would also be useful to extend the approach to articulated models, and take into account self-collisions between various links.
In order to handle deformable models, we would like to develop incremental techniques that can refine the contact
space approximation for deformable objects.
\item[Configuration Space Construction using Instance-based Learning] First, we need to find methods to adjust LSH parameters adaptively so that the $\knn$ query becomes more efficient for varying dataset sizes. One possible approach is to change $L$ (the number of hash tables), because a small $L$ may provide sufficient $\knn$ candidates for a large dataset. Second, for samples in regions that are well-explored, we should avoid inserting their collision results into the dataset in order to limit the dataset size. Moreover, since prior collision results are stored in hash tables, we can efficiently update the data without high overhead. Thus, we can extend the instance-based learning framework to improve the performance of planning algorithms in dynamic environments. Lastly, we would like to evaluate performance in dynamic scenes.
\item[Parallel Motion Planning Framework] Our current parallel planning work is limited to the PRM planning algorithm, but it is possible to extend it for accelerating other widely-used planning algorithms, including RRT or optimization-based planners. In addition, we are interested in extending GPU planning algorithms to high-DOF articulated models. We are also interested in using exact algorithms for local planning. Moreover, we hope to apply our real-time algorithms to dynamic scenarios.
\item[Parallel Collision Detection] We are interested in using more advanced sampling schemes with the GPU-based planner, to further improve its performance and deal with narrow passages. Furthermore, we would like to
modify the planner to generate smooth paths that take into account kinematic and dynamic constraints.
\item[Parallel $k$-Nearest Neighbor] We hope to test our algorithm on more real-world datasets, including images, videos, and so forth. We also need to design efficient out-of-core algorithms to handle very large datasets (e.g., $>$ 100GB), as the on-chip memory on a GPU is limited to a few GBs. We need to further analyze the quality of our bi-Level scheme on large spatial databases.

\item[Proximity Computations on Noisy Sensor Data] We need to test the performance of our algorithm on different robotic systems and evaluate its performance
on tasks such as planning and grasping. It would be useful to extend this approach to continuous collision checking, which takes into account the motion of the robot
between discrete intervals along its path. Similar probabilistic methods can also be developed for other queries,
including separation and penetration depth computation. Finally, we are interested in improving the algorithm to handle dynamic environments, where points may change position or can be added or removed from the environment due to movement, occlusion, or incremental data.
The algorithm for dynamic environments should handle incremental data efficiently and may benefit from various incremental techniques including incremental SVM~\cite{Gert:nips:2001} and BVH refitting techniques~\cite{Lauterbach10}.
\item[Proximity Computation for Streaming Noisy Sensor Data] We are interested in further improving the collision
checking and distance query implementations. We are also interested in
applications of this work to motion planning and active sensing. For example, we would like
to design strategies for gaining more information about uncertain or
unknown parts of the environment.
\end{description}






